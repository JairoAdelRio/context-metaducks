\usemodule[mag-01]
\usemodule[metaducks]
\startdocument
  [title={Metaducks},
   subtitle={Ducks for \CONTEXT !},
   author={Jairo A. del Rio},
   date=\currentdate,
  ]

\startsubject[title=Acknowledgements]

Although everybody knows an acknowledgement list will always be incomplete, I want to mention some of the people who made this possible:

Sam Carter for the original idea and the nice package she created and maintains. Paulo Cereda, whose suggestion avoided this module to be called a plain \quote{\type{ducks}} or a more boring possibility. Hans Hagen, Wolfgang Schuster, Aditya Mahajan, Henri Menke and the \CONTEXT\ user base for answering my questions and giving tips and tricks to properly use \CONTEXT\ with its unbeatable tools. And all \TEX\ users, no matter if Plain, \LATEX, \CONTEXT\ or another, for such a great and diverse community. All duck lovers around the world, regardless of language, religion or nationality. Ducks rock (and swim)! For all of you is this module.

\stopsubject

\startsubject[title=Rationale]

I love ducks as much as I love my girlfriend, who is a duck breeder, and almost as much I'm abhorred by TikZ's feature creep, its slowness and its huge manual. Don't misunderstand me, I hugely appreciate the effort users and maintainers put on TikZ and even desire some features/libraries to be implemented in \CONTEXT, such as \type{graphdrawing}; however, I still don't like TikZ. Besides that, TikZ is rather geared to \LATEX\ and keeping things right with \CONTEXT\ sometimes is a toothache. I think some \TEX\ users who migrate from \LATEX\ to \CONTEXT\ or just like ducks and \CONTEXT, but are uncomfortable with TikZ, would be pleased with this little module. 

\stopsubject

\startsubject[title=Differences and caveats]

Because this is a port from a package written in TikZ, it's worthwhile to explain differences and limitations with respect to \LATEX's TikZducks. There is, for instance, a difference between  PGF/TikZ and \CONTEXT\ way of handling with key-value pairs. \type{pgfkeys} is very happy with the following:

\starttyping
\somecommand[option1,key1=value1,key2=value2,option2]
\stoptyping

\CONTEXT, on the other hand, doesn't usually mix those alternatives, and in order not to overcomplicate this module (after all, \MetaPost\ gives faster results when cautiously exploited), this module is based on a rather straightforward key-value interface. For instance, \type{body=\MPcolor{blue}} won't work and you should use \type{bodycolor=\MPcolor{blue}} instead. The same applies for eyes, bill and head. Another instance: if you need a crozier for your duck, the way to specify is

\starttyping
\ducks
    [
    crozier=true, %Don't forget this!
    croziercolor={.5[\MPcolor{orange},black]}
    ]
\stoptyping

If you only set \type{crozier=true}, a default color will be selected.

Additionally, colors should be passed in a \METAPOST-understandable way, e.g. \type{\MPcolor{brown}}, \type{1} (RGB white) or \type{(0.5,0.4,1)}. \type{xcolor}'s color mixing isn't supported, but remember you're still able to define colors in such a way with \type{\enabledirectives[colors.pgf]} in your preamble. Like this:

\starttyping
\enabledirectives[colors.pgf]
\usecolors[svg]
\usemodule[metaducks]
%Only for two colors!
\definecolor[mycolor][gold!50!violet]
\starttext
\ducks[bodycolor=\MPcolor{mycolor}]
\stoptext
\stoptyping

Surely a cleverer alternative is possible and I'll have to find out it (possibly guided by Hans and Wolfgang). In the meantime those are your options.

Another point on color: this module includes a companion called \type{xcolor} (actually \type{colo-imp-xcolor.mkiv}) so ducks are roughly the same both in \CONTEXT\ and \LATEX. Because of this, we're actually using colors named \type{xcoloryellow}, \type{xcolorcyan} and so on to avoid clashes with other color schemes in \CONTEXT. So, you should be able of using something like

\starttyping
\usecolors[xwi]
\ducks[bodycolor={\MPcolor{gold}}]
\stoptyping

and keep your ducks of the same color. 

As for size, and unlike \MetaPost, TikZ defaults to centimeters. To adjust that, I've included an \type{unitsize} key in order to scale your picture accordingly. Just ensure to pass an unit Metafun/\MetaPost\ can interpret correctly. 

Most of TikZducks options are available, but don't forget to initialize them via \type{<option>=true} and only then \type{<option>color=...}, when possible. Random ducks are also available. Obviously you should be familiar with TikZducks manual to see all available options (or you could look at \type{t-metaducks.mkvi}, too).

So far, stripes aren't implemented (but I promise I'll do soon).

\stopsubject

\startsubject[title=Some examples]

\startbuffer
\ducks
\stopbuffer

\typebuffer \getbuffer

\startbuffer
\ducks
[
unitsize=1.4142cm,
signpost=true,
bodycolor={(.4,0.1,0.7)},
signtext=\TEX
]
\stopbuffer

\typebuffer \getbuffer

\startbuffer
\randomducks[bodycolor=darkblue]
\stopbuffer

\typebuffer \getbuffer

\startbuffer
\randomducks[speech=true,bubblecolor=red,speechtext=Hey!]
\stopbuffer

\typebuffer \getbuffer

\startbuffer
\ducks[snowduck=true,snowduckcolor=.9]
\stopbuffer

\typebuffer \getbuffer

And a gift for Christmas or whatever you celebrate in December:

\startbuffer
\ducks[unitsize=2cm,santa=true,mug=true]
\stopbuffer

\typebuffer \getbuffer

\stopsubject

\startsubject[title=To do]

\startitemize[packed]
\startitem A better manual. \stopitem
\startitem Support for stripes. \stopitem
\startitem Further customization. \stopitem
\startitem Easier color specification. \stopitem
\startitem IMPORTANT: Acess from MP environments. \stopitem
\startitem Complete this list (!).\stopitem
\stopitemize

\stopsubject

\stopdocument